\documentclass{article}
\usepackage{multicol}
\usepackage[margin=0.5in]{geometry}
\usepackage{amssymb}
\usepackage{enumerate}
\usepackage{amsmath}
\usepackage{mathtools}
%\usepackage{macros}
\usepackage{fancyvrb}
\usepackage{etoolbox}
\makeatletter
\preto{\@verbatim}{\topsep=0pt \partopsep=0pt}
\makeatother
\DefineVerbatimEnvironment{code}{Verbatim}{fontsize=\small}
\DefineVerbatimEnvironment{example}{Verbatim}{fontsize=\small}

\begin{document}
\scriptsize
\begin{multicols}{3}
  \raggedright
  {\bf Files and Directories}

  {\tt int open(char* pathname, int flags)}. Some useful flags:
  {\tt O\_WRONLY, O\_RDONLY, O\_CREAT, O\_EXCL}. Returns file descriptor. -1 on
  error.

  {\tt DIR* opendir(char* path)}.\\
  {\tt struct dirent* readdir(DIR* dir)}. \texttt{struct dirent*} represents an
  entry (file or subdirectory) inside a directory:
  \begin{verbatim}
struct dirent {
  ino_t d_ino;    // inode number
  char  d_name[]; // name of entry
};\end{verbatim}
  {\tt ssize\_t read(int fd, void* buf, size\_t count)} Read up to {\tt count}
  bytes from the fd into buf.

  {\tt ssize\_t write(int fd, const void* buf, size\_t count)} write up to {\tt
  count} bytes into the fd from buf.

  {\tt int close(fd)} returns 0 on success.

  {\tt stat} populates a {\tt struct stat}:
  {\tt S\_ISREG, S\_ISDIR, S\_ISLINK(st.st\_mode) }.
  {\tt stat(char* path, struct stat *st)} populates \texttt{struct stat} defined
  as
{\tiny
  \begin{verbatim}
struct stat {
  dev_t     st_dev;     /* ID of device containing file */
  ino_t     st_ino;     /* inode number */
  mode_t    st_mode;    /* protection */
  nlink_t   st_nlink;   /* number of hard links */
  uid_t     st_uid;     /* user ID of owner */
  gid_t     st_gid;     /* group ID of owner */
  dev_t     st_rdev;    /* device ID (if special file) */
  off_t     st_size;    /* total size, in bytes */
  blksize_t st_blksize; /* blocksize for file system I/O */
  blkcnt_t  st_blocks;  /* number of 512B blocks allocated */
  time_t    st_atime;   /* time of last access */
  time_t    st_mtime;   /* time of last modification */
  time_t    st_ctime;   /* time of last status change */
}; \end{verbatim}
}
  {\tt lstat} Same as {\tt stat} except {\tt lstat} returns info about the
  link while {\tt stat} returns info about the linked file if a file is a link.

  {\tt int dup2(int oldfd, int newfd)} makes new fd point to oldfd, closing new
  if necessary. For example, {\tt dup2(fd, STDIN\_FILENO)} to dump fd into STDIN.

  \noindent\rule{4cm}{0.4pt}

  {\bf Layering and Naming}
  %TODO: look at book

  \noindent\rule{4cm}{0.4pt}

  {\bf v6 Data structures}
  %TODO: look at project handout

  \noindent\rule{4cm}{0.4pt}

  {\bf Processes}

  {\tt pit\_t fork()} called once, returns twice. {\tt pid = 0} if in child
  process.

  {\tt waitpid(pid\_t pid, int *status, int options)} store status info into\\
  {\it status}: {\tt WIFEXITED(status), WEXITSTATUS, WIFSIGNALED, WTERMSIG, WCOREDUMP,
  WIFSTOPPED} need untraced, {\tt WSTOPSIG} number of signal that stopped the
  child (if ifstopped returned true), {\tt WIFCONTINUED} child resumed by
  sigcont.\\
  {\it Pid}: $<-1$ (wait for any child process whose process group ID is equal to the
  absolute value of pid.), $-1$ (wait for any child process), $0$ (wait for any
  child process whose process group ID is equal to that of the calling process),
  $>0$ (wait for that specific pid)\\
  {\it Options}: {\tt WNOHANG} (return immediately if no child has exited), {\tt
  WUNTRACED} (return if a child has stopped even if not traced by ptrace) {\tt
  WCONTINUED} (return if a stopped child has been resumed by delivery of
  SIGCONT).

  {\tt int execvp(const char* path, char* argv[])} Consumes the process.
\begin{verbatim}
pid_t pid = fork();
exitIf(pid == -1, kForkFailed,
  stderr, "fork function failed.\n");
if (pid == 0) {
  if (execvp(argv[0], argv) < 0) {
    printf("%s: Command not found\n",
      argv[0]);
    exit(0);
  }
}\end{verbatim}

  {\tt int signal(int signum, (void sighandler(int signum)))}. If process gets
  {\tt signum}, it calls the sighandler. {\tt SIGCHLD}\\
  {\it Signal handlers} are functions that return voids. Need to use global (no
  state).
  \begin{verbatim}
static void reapChild(int sig) {
pid_t pid;
while (true) {
  pid = waitpid(-1, NULL, WNOHANG);
  if (pid <= 0) break;
  numChildrenDonePlaying++;
} \end{verbatim}
  {\it Signal blocking} Useful for handling SIGCHLDs etc only after they, as
  jobs, have been added to a list or something, so proper concurrency can
  happen.
  \begin{verbatim}
  sigset_t mask;
  sigemptyset(&mask);
  sigaddset(&mask, SIGCHLD);
  sigprocmask(SIG_BLOCK, &mask, NULL);
  // critical region
  sigprocmask(SIG_UNBLOCK,&mask,NULL);\end{verbatim}

  {\tt int kill(pid\_t pid, int sig)} Sends sig to process pid. 0 success, -1
  error.

  {\it Process ids} %TODO?

  {\it Process groups} A parent will create a group with its child -- so can
  send signals to all of its children.

  {\it Interprocess Concurrency} Use blocking to handle signals only when
  possible. Use pipes to handle communications.

  \noindent\rule{4cm}{0.4pt}

  {\bf Threading}
  Threads have independent stacks, but they all share access to the same text,
  data, and heap segments. Stack segment is subdivided into multiple miniature
  stacks. Thread manager time slices and switches between simultaneously running
  threads in much the same way that the kernel scheduler switches between
  processes.
  {\it Pro}\\
  - Easier to support communication between threads, because address spaces
  accessible are largely the same.\\
  - Multiple threads can access the same global data and one copy of the code.\\
  - One thread can share its stack space (via pointers and references) with
  another thread.\\
  {\it Con}\\
  - Multiple threads can access the same global data and one copy of the code.\\
  - One thread can share its stack space (via pointers and references) with
  another thread.

  {\tt pthread\_t pthread\_create(pthread\_t *pt, const pthread\_attr\_t *attr,
  void* (routine)(void* args))}. Can pass arguments in the void *args.\\
  {\tt pthread\_join(pthread\_t thread, void **retval)} waits for thread to
  terminate. Returns 0 on success, else returns error number. Copies return
  value into void **retval.

  {\it Race conditions}

  {\it Critical regions}

  {\tt thread t = thread((void *) (thread function)(arg1, arg2, \ldots))} Use
  {\tt t.detach()} to detach thread and have it run on its own. Use {\tt
  t.join()} to join.

  \noindent\rule{4cm}{0.4pt}

  {\bf Locking}

  {\tt mutex m}, {\tt m.lock()}, {\tt m.unlock()}

  {\tt lock\_guard<mutex> lg(m)}. Constructor locks it. Destructor unlocks it.

  {\tt unique\_lock<mutex> ul(m, defer\_lock)}. Wraps around anything with
  lock/unlock. Used for condition variables.

  {\tt condition\_variable cv}, {\tt cv.wait(ul, bool (function()))}. Checks if
  function returns true. If false, unlocks ul, waits to be notified to check
  function again. {\tt cv.notify\_all()} notifies every instance of cv that is
  waiting.

  {\tt semaphore s(int value).} Atomic counter of a shared resource. No
  get-value since value could change. No copy constructor. {\tt
  s.signal()} signals resource added (++), {\tt s.wait()} blocks until there is
  at least one variable. No busy waiting!

  {\it Rendezvousing etc\ldots} Reader/Writer example.
  \begin{verbatim}
static const unsigned int kNumBuffers = 30;
static const unsigned int kNumCycles = 4;

static char buffer[kNumBuffers];
static semaphore emptyBuffers(kNumBuffers);
static semaphore fullBuffers(0);

static void writer() {
  for (unsigned int i = 0;
  i < kNumCycles * kNumBuffers; i++) {
    char ch = prepareData();
    // don't try to write to a
    // slot unless you know it's empty
    emptyBuffers.wait();
    buffer[i % kNumBuffers] = ch;
    // signal reader more to read
    fullBuffers.signal();
  }
}

static void reader() {
  for (unsigned int i = 0;
  i < kNumCycles * kNumBuffers; i++) {
    // don't try to read from a
    // slot unless you know it's full
    fullBuffers.wait();
    char ch = buffer[i % kNumBuffers];
    // signal writer there is open slot for data
    emptyBuffers.signal();
    processData(ch);
  }
}

int main(int argc, const char *argv[]) {
  thread w(writer);
  thread r(reader);
  w.join();
  r.join();
  return 0;
}
  \end{verbatim}

  \noindent\rule{4cm}{0.4pt}

  {\bf Networking}

  {\it Client-server model} Servers wait by a phone (socket), bound to a port,
  and listens for incomming requests to accept as they come in.

  {\it {\tt gethostbyname} vs. {\tt gethostbyname\_r}}
  {\tt struct hostent *gethostbyname(const char *name)} feed in
  ``www.google.com'' to get a hostent. %TODO r?

  {\tt int getaddrinfo(const char *hostname, const char *servname,
  const struct addrinfo *hints, struct addrinfo **res);
  void freeaddrinfo(struct addrinfo *ai);} implants the head of a
  linked list of struct addrinfo records in the space addressed by
  res.\\
  {\tt freeaddrinfo(struct addrinfo *ai)} fully disposes of the linked list
  surfaced by an earlier call to getaddrinfo.\\
  {\tt int getnameinfo(const struct sockaddr *sa, socklen\_t salen,
  char *host, socklen\_t hostlen,
  char *serv, socklen\_t servlen, int flags);} gets ip address info.

  {\tt struct hostent *gethostbyaddr(const char *addr, int len, int type);}

  {\tt sock\_addr}

  {\bf Client code}
  \begin{verbatim}
int createClientSocket(
const string& host,
unsigned short port) {
  struct hostent *he =
    gethostbyname(host.c_str());
  if (he == NULL)
  return kClientSocketError;
  int s=socket(AF_INET,SOCK_STREAM,0);
  if (s < 0)
    return kClientSocketError;
  struct sockaddr_in serverAddress;
  memset(&serverAddress, 0,
    sizeof(serverAddress));
  serverAddress.sin_family = AF_INET;
  serverAddress.sin_port = htons(port);
  serverAddress.sin_addr.s_addr =
    ((struct in_addr *)
      he->h_addr)->s_addr;
  if (connect(s, (struct sockaddr *)
    &serverAddress,
    sizeof(serverAddress)) != 0) {
    close(s);
    return kClientSocketError;
  }
  return s;
}

int main(int argc, char *argv[]) {
  int clientSocket =
    createClientSocket(
    "myth22.stanford.edu", 12345);
  if (clientSocket == kClientSocketError) {
    cerr << "Time server
      could not be reached" << endl;
    cerr << "Aborting" << endl;
    return kTimeServerInaccessible;
  }
  sockbuf sb(clientSocket);
  iosockstream ss(&sb);
  string timeline;
  getline(ss, timeline);
  cout << timeline << endl;
  return 0;
}
  \end{verbatim}

  {\bf Server code}
  \begin{verbatim}
static const int kReuseAddresses = 1;
int createServerSocket(
unsigned short port, int backlog) {
  int serverSocket = socket(AF_INET,
    SOCK_STREAM, 0);
  if (serverSocket < 0)
    return kServerSocketFailure;
  if (setsockopt(serverSocket,
    SOL_SOCKET, SO_REUSEADDR,
    &kReuseAddresses,
    sizeof(int)) < 0) {
      close(serverSocket);
      return kServerSocketFailure;
  }
// IPv4-style socket address
  struct sockaddr_in serverAddress;
  memset(&serverAddress, 0,
    sizeof(serverAddress));
// sin_family field used to
// self-identify sockaddr type
  serverAddress.sin_family = AF_INET;
  serverAddress.sin_addr.s_addr =
    htonl(INADDR_ANY);
  serverAddress.sin_port = htons(port);
// bind the unbound socket
// to the (address, port) pair
  if (bind(serverSocket,
    (struct sockaddr *) &serverAddress,
    sizeof(struct sockaddr_in)) < 0) {
      close(serverSocket);
      return kServerSocketFailure;
  }
// listen through the server socket, to
// allow a backlog of the specified size
  if (listen(serverSocket, backlog) < 0) {
    close(serverSocket);
    return kServerSocketFailure;
  }

  return serverSocket;
}
  \end{verbatim}

  \noindent\rule{4cm}{0.4pt}

  {\bf Socket API calls}

  {\tt int socket(int socket\_family, int socket\_type, int protocol)}

  {\tt int bind(int sockfd, const struct sockaddr *addr, socklen\_t addrlen)}
  assigns the address specified by addr to the socket fd.

  {\tt int listen(int sockfd, int backlog)} marks sockfd as passive, accepting
  incoming connection requests with accept. Backlog is the maximum length of the
  queue of waiting requests.

  {\tt accept}

  \noindent\rule{4cm}{0.4pt}

  {\bf Networking Examples}

  {\tt popen}

  {\tt pclose}

  \noindent\rule{4cm}{0.4pt}

  {\bf HTTP requests}

  {\it Request line}

  {\it Header}

  \noindent\rule{4cm}{0.4pt}

  {\bf MapReduce}

\end{multicols}
\end{document}
